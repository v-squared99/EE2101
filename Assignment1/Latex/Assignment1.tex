\documentclass[journal,12pt,twocolumn]{IEEEtran}

\usepackage{setspace}
\usepackage{gensymb}

\singlespacing


\usepackage[cmex10]{amsmath}

\usepackage{amsthm}

\usepackage{mathrsfs}
\usepackage{txfonts}
\usepackage{stfloats}
\usepackage{bm}
\usepackage{cite}
\usepackage{cases}
\usepackage{subfig}

\usepackage{longtable}
\usepackage{multirow}

\usepackage{enumitem}
\usepackage{mathtools}
\usepackage{steinmetz}
\usepackage{tikz}
\usepackage{circuitikz}
\usepackage{verbatim}
\usepackage{tfrupee}
\usepackage[breaklinks=true]{hyperref}
\usepackage{graphicx}
\usepackage{tkz-euclide}
\usetikzlibrary{shapes,arrows}


\usetikzlibrary{calc,math}
\usepackage{listings}
    \usepackage{color}                                            %%
    \usepackage{array}                                            %%
    \usepackage{longtable}                                        %%
    \usepackage{calc}                                             %%
    \usepackage{multirow}                                         %%
    \usepackage{hhline}                                           %%
    \usepackage{ifthen}                                           %%
    \usepackage{lscape}     
\usepackage{multicol}
\usepackage{chngcntr}

\DeclareMathOperator*{\Res}{Res}

\renewcommand\thesection{\arabic{section}}
\renewcommand\thesubsection{\thesection.\arabic{subsection}}
\renewcommand\thesubsubsection{\thesubsection.\arabic{subsubsection}}

\renewcommand\thesectiondis{\arabic{section}}
\renewcommand\thesubsectiondis{\thesectiondis.\arabic{subsection}}
\renewcommand\thesubsubsectiondis{\thesubsectiondis.\arabic{subsubsection}}


\hyphenation{op-tical net-works semi-conduc-tor}
\def\inputGnumericTable{}                                 %%

\lstset{
%language=C,
frame=single, 
breaklines=true,
columns=fullflexible
}

\begin{document}
\tikzstyle{block} = [draw, rectangle, 
    minimum height=0.5em, minimum width=1em]
\tikzstyle{sum} = [draw, circle, node distance=1cm]
\tikzstyle{input} = [coordinate]
\tikzstyle{output} = [coordinate]
\tikzstyle{pinstyle} = [pin edge={to-,thin,black}]

\newtheorem{theorem}{Theorem}[section]
\newtheorem{problem}{Problem}
\newtheorem{proposition}{Proposition}[section]
\newtheorem{lemma}{Lemma}[section]
\newtheorem{corollary}[theorem]{Corollary}
\newtheorem{example}{Example}[section]
\newtheorem{definition}[problem]{Definition}

\newcommand{\BEQA}{\begin{eqnarray}}
\newcommand{\EEQA}{\end{eqnarray}}
\newcommand{\define}{\stackrel{\triangle}{=}}
\bibliographystyle{IEEEtran}
\providecommand{\mbf}{\mathbf}
\providecommand{\pr}[1]{\ensuremath{\Pr\left(#1\right)}}
\providecommand{\qfunc}[1]{\ensuremath{Q\left(#1\right)}}
\providecommand{\sbrak}[1]{\ensuremath{{}\left[#1\right]}}
\providecommand{\lsbrak}[1]{\ensuremath{{}\left[#1\right.}}
\providecommand{\rsbrak}[1]{\ensuremath{{}\left.#1\right]}}
\providecommand{\brak}[1]{\ensuremath{\left(#1\right)}}
\providecommand{\lbrak}[1]{\ensuremath{\left(#1\right.}}
\providecommand{\rbrak}[1]{\ensuremath{\left.#1\right)}}
\providecommand{\cbrak}[1]{\ensuremath{\left\{#1\right\}}}
\providecommand{\lcbrak}[1]{\ensuremath{\left\{#1\right.}}
\providecommand{\rcbrak}[1]{\ensuremath{\left.#1\right\}}}
\theoremstyle{remark}
\newtheorem{rem}{Remark}
\newcommand{\sgn}{\mathop{\mathrm{sgn}}}
%\providecommand{\hilbert}{\overset{\mathcal{H}}{ \rightleftharpoons}}
\providecommand{\system}{\overset{\mathcal{H}}{ \longleftrightarrow}}
    %\newcommand{\solution}[2]{\textbf{Solution:}{#1}}
\newcommand{\solution}{\noindent \textbf{Solution: }}
\newcommand{\cosec}{\,\text{cosec}\,}
\providecommand{\dec}[2]{\ensuremath{\overset{#1}{\underset{#2}{\gtrless}}}}
\newcommand{\myvec}[1]{\ensuremath{\begin{pmatrix}#1\end{pmatrix}}}
\newcommand{\mydet}[1]{\ensuremath{\begin{vmatrix}#1\end{vmatrix}}}
\numberwithin{equation}{subsection}
\makeatletter
\@addtoreset{figure}{problem}
\makeatother
\let\StandardTheFigure\thefigure
\let\vec\mathbf
\renewcommand{\thefigure}{\theproblem}
\def\putbox#1#2#3{\makebox[0in][l]{\makebox[#1][l]{}\raisebox{\baselineskip}[0in][0in]{\raisebox{#2}[0in][0in]{#3}}}}
     \def\rightbox#1{\makebox[0in][r]{#1}}
     \def\centbox#1{\makebox[0in]{#1}}
     \def\topbox#1{\raisebox{-\baselineskip}[0in][0in]{#1}}
     \def\midbox#1{\raisebox{-0.5\baselineskip}[0in][0in]{#1}}
\vspace{3cm}




\title{EE2101-Control Systems Assignment 1}
\author{Vinjam Lakshmi Sai Vignatha \\ Es17btech11024}
\maketitle
\newpage
\bigskip
\renewcommand{\thefigure}{\theenumi}
\renewcommand{\thetable}{\theenumi}
Download codes from 
\begin{lstlisting}
https://github.com/v-squared99/EE2101/tree/master/Assignment1
\end{lstlisting}
\begin{abstract}
This document contains the solution to problem 56 from chapter 2 of \textbf{Control Systems Engineering by Norman Nise}
\end{abstract}

\section{Problem}
For a given differential equation model: 
\begin{align}
\frac{dC(t)}{dt} = -(\lambda + \mu + \gamma + \delta + \nu)C(t) + \lambda N(t)
\\\frac{dN(t)}{dt} = -(\nu + \delta)C(t) - \mu N(t) + I(t)
\end{align}
where
\begin{align}
\nu = \delta = 0.05, \mu = 0.02, \gamma = 0.08, \lambda = 0.07
\\ C(0) = C_o = 47,000,500
\\N(0) = N_o = 61,100,500 
\\I(t) = I = 6 x 10^6
\end{align}
(a) A block diagram is to be drawn showing the output N(s), the input I(s), the transfer function, and the initial conditions.\newline(b) Analytic expression for N(t) for $t \geq 0$ is to be found using any method. 


\section{Solution}
\subsection{Part(a)}
After substituting the values in (1.0.1) and (1.0.2), we get:
\begin{align}
\frac{dC(t)}{dt} = -0.9C(t) + 0.7N(t)
\\\frac{dN(t)}{dt} = -0.1C(t) - 0.02N(t) + I(t)
\end{align}
Finding the Laplace Transform of the above equations:
\begin{align}
sC(s) - C(0) = -0.9C(s) + 0.7N(s)
\\sN(s) - N(0) = -0.1C(s) - 0.02N(s) + I(s)
\end{align}
Substituting C(0) and N(0) gives us:
\begin{align}
C(s)(s + 0.9) = 0.7N(s) + 47000500
\\N(s)(s + 0.02) = -0.1C(s) + I(s) + 61100500
\end{align}
Substituting (2.0.5) in (2.0.6):
\begin{align}
N(s)= \frac{\frac{I(s)+61100500}{s+0.02} + \frac{-47000500}{(s+0.02)(s+0.9)}}{1+\frac{0.07}{(s+0.02)(s + 0.9)}}
\\ = \frac{(I(s)+61100500)(s + 0.9) - 47000500}{s^2 + 0.92s + 0.088}
\\\implies N(s) = \frac{I(s)+ \frac{61100500s + 7989950}{s+0.9}}{s^3 + 1.82s^2 + 0.916s + 0.0792}
\end{align}

\begin{tikzpicture}[auto, node distance=2cm,>=latex']
    \node [input, name=input] {};
    \node [sum, right of=input] (sum) {};
    \node [block, right of=sum] (controller) {$G(s)$};
    \node [block, below of=controller] (feedback) {$H(s)$};
    \node [output, right of=controller] (output) {};
    
    \draw [draw,->] (input) -- node {$I(s)$} (sum);
    \draw [->] (controller) -- node [name=y] {$N(s)$}(output);
    \draw [->] (sum) -- node {} (controller);
    
    \draw [->] (feedback) -| node[pos=0.99] {$+$} 
        node [near end] {} (sum);
    \draw [->] (y) |- (feedback);
\end{tikzpicture}



\subsection{Part(b)}
We know that
\begin{align}
    I(s) = \frac{6 \times 10^6}{s}
\end{align}
Substituting (2.2.1) in (2.1.8):
\begin{align}
    N(s) = \frac{( \frac{6 \times 10^6}{s}+61100500)(s + 0.9) - 47000500}{s^2 + 0.92s + 0.088}
\\\implies N(s) = \frac{61100500s^2 + 56290400s + 54\times10^5 }{s(s + 0.8116)(s + 0.1084)}
\end{align}
Solving this using Partial Fractions:
\begin{align}
    N(s) = \frac{A}{s} + \frac{B}{s+0.8116} + \frac{C}{s+0.1084}
\end{align}
Solving for A, B and C, we get:
\begin{align}
    A = 61.364 \times 10^6
   \\B = -7\times10^4
\\C = -19.3\times10^4
\end{align}
Inverse Laplace Transform of N(s) equation is:
\begin{align}
    N(t) = 6.1364 \times 10^7 -7\times10^4 e^{-0.8116t} -19.3\times10^4 e^{-0.1084t}
\end{align}

\end{document}





